    \section{Research Experience}

        \begin{twocolentry}{
        Beijing, China   \\
        Jan 2026 –  Aug 2026}
            \textbf{Research Intern}, Microsoft Research
            
            \textbf{Mentor:} Dr. Jing Liu
        \end{twocolentry}

        \begin{onecolentry}
            \textit{Joining Systems Research Group }
        \end{onecolentry}


        \begin{twocolentry}{
        Champaign, IL   \\
        May 2025 –  Present}
            \textbf{Research Assistant}, UIUC\&Princeton
            
            \textbf{Mentor:} Prof. Francis Y. Yan \& Prof. Ravi Netravali 
        \end{twocolentry}

        \begin{onecolentry}
            \textit{Real-time vision language model streaming}
            \begin{highlights}
                \item Profiled and measured proprietary and open-sourced real-time vision language models.
                \item Co-optimizing real-time communication gateway and inference backends.
                \item Improving real-time user experience with long visual context support using video RAG.
            \end{highlights}
        \end{onecolentry}

        \begin{onecolentry}
            \textit{Multi-party video conferencing}
            \begin{highlights}
                \item Built an auto-scaling prototype for the Selective Forwarding Unit (SFU) based on Jitsi and K8s.
                \item Prototyped an XDP ingress accelerated SFU.
                \item Building open-sourced auto-scaling, cloud native, modern SFU at Pion using modern Golang Webrtc.
            \end{highlights}
        \end{onecolentry}


        
        \begin{twocolentry}{
        UIUC, Champaign, IL   \\
        May 2024 –  May 2025}
            \textbf{Research Assistant}, UIUC\&Akamai
            
            \textbf{Mentor:} Prof. Deepak Vasisht \& Prof. Ramesh K. Sitaraman
        \end{twocolentry}

        \begin{onecolentry}
            \textit{Satellite based CDN }\ref{pub:starcdn}
            \begin{highlights}
                \item Proposed StarCDN, a novel LEO satellite-based CDN architecture that achieves low latency and high bandwidth usage via consistent hashing and orbit-aware relayed fetch.
                \item Achieved 15\% hit rate improvement against baseline LRU design.
                \item Independently developed and open-sourced, in collaboration with Akamai, a CDN synthetic trace generator for geographically diverse traffic with theoretically proven cache-level and object-level properties.
                \item Trace generator achieved sub 2\% difference in caching characteristics.
            \end{highlights}
        \end{onecolentry}

        \begin{twocolentry}{
        Champaign, IL   
            
        May 2023 - Aug 2024}
            \textbf{Research Assistant}, UIUC\\
            \textbf{Mentor:} Prof. Tianyin Xu
        \end{twocolentry}

        \begin{onecolentry}
            \textit{Cloud Management System Interaction }\ref{pub:watcher}
            \begin{highlights}
                \item Studied 13 open-sourced K8s operators, and 412 reported failures to identify challenges in cloud operator reliability.
                \item Helped build a small tool to identify 86 new bugs that cause failure between the operator and the managed system.
            \end{highlights}
            \textit{Cloud and Emulator Discrepancies} \ref{pub:cloudtest}
            \begin{highlights}
                \item Reasoned fundamental challenges in building reliable cloud service emulators via studying 10 real-world cloud-based applications and fuzzing 255 Azure and AWS APIs.
                \item Identified discrepancies in 37\% of cloud storage APIs and dissected the root causes and manifestations into 8 findings.
                \item Contributed to a testing proxy middleware that automatically detects discrepancies between cloud and emulator APIs, which carries out CI workflows with high fidelity while minimizing costs.
                
            \end{highlights}
        \end{onecolentry}
